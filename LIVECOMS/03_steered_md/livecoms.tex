\subsubsection{Introduction}
Many relevant biological processes, such as transmembrane permeation or transitions between active and inactive protein conformations, occur on a timescale of microseconds to seconds ~\cite{Zwier2010,Choy2017,Wells2007}. However, even with GPU acceleration, the timescales accessible via MD simulations are only a few hundred ns/day ~\cite{HecBioSim_benchmark}. One of the methods to get around this limitation is steered molecular dynamics (sMD). sMD involves applying a harmonic restraint to bias the system towards a conformation defined through one or more collective variables (CVs):

\begin{equation}
V(\vec{s},t) = \frac{1}{2} \kappa(t) ( \vec{s} - \vec{s}_0(t) )^2,
\label{eq:sMD}
\end{equation}

where $\kappa$ is the force constant, $\vec{s}_0$ is the expected CV value at a specific timestep, and $\vec{s}$ is the actual CV value at that timestep \cite{Isralewitz2001,Tribello2014}.

This section of the tutorial summarizes how to use BioSimSpace to set up and run sMD simulations. BSS prepares input files for PLUMED, which is the software that works together with MD engines such as AMBER and GROMACS to add the restraint in eq~\ref{eq:sMD}.

We use protein tyrosine phosphatase 1B (PTP1B) as the system of choice for this tutorial. It is a negative regulator of insulin signalling ~\cite{sMD_ptp1b-diabetes}, and is an attractive target for type II diabetes ~\cite{sMD_Wiesman}. The function of PTP1B depends on the conformation of its WPD loop, which can be closed (active) or open (inactive) (Figure~\ref{fig:ptp1b}). The WPD loop of PTP1B opens and closes on a $\mu$s timescale ~\cite{Choy2017}, and therefore this transition is not observed on conventional computational timescales.

\begin{figure}[htp]
\includegraphics[width=\linewidth]{LIVECOMS/03_steered_md/open-close.png}
\caption{The WPD loop of PTP1B, in two conformations: open (yellow, PDB ID: 2HNP) and closed (red, PDB ID: 1SUG).}
\label{fig:ptp1b}
\end{figure}

\subsubsection{Running sMD using BioSimSpace}

The first notebook of this tutorial \href{https://github.com/OpenBioSim/biosimspace_tutorials/blob/main/03_steered_md/01_setup_sMD.ipynb}{01-setup-sMD} describes how to set up a steered MD simulation with BioSimSpace. 
The notebook illustrates the use of \href{https://biosimspace.openbiosim.org/api/generated/BioSimSpace.Metadynamics.CollectiveVariable.RMSD.html#BioSimSpace.Metadynamics.CollectiveVariable.RMSD}{BSS.Metadynamics.CollectiveVariable.RMSD} to define a collective variable that enforces a conformational change of the 'WPD' loop in the enzyme PTP1B. 
Next \href{https://biosimspace.openbiosim.org/api/generated/BioSimSpace.Protocol.Steering.html#BioSimSpace.Protocol.Steering}{BSS.Protocol.Steering} is used to specify a steering schedule alongside the RMSD CV. The notebook illustrates how input files for the \emph{gmx}, \emph{sander} or \emph{pmemd} MD engines can be subsequently prepared. The notebook also shows how a more complex steering schedule that combines multiple CVs can be written. 

For production simulations, we recommend long sMD simulations to minimize the strength of the bias that needs to be applied to enforce the desired conformational change by the end of the steered MD simulation. Optimizing the steered MD schedule parameters requires trial and error. 
For convenience, we provide simple Python scripts with a command line interface to execute steered MD runs and scan schedule parameters for two specified sMD protocols (\href{https://github.com/OpenBioSim/biosimspace_tutorials/blob/main/03_steered_md/scripts/sMD_simple.py}{a single CV} and a \href{https://github.com/OpenBioSim/biosimspace_tutorials/blob/main/03_steered_md/scripts/sMD_multiCV.py}{multiple CV} example). We also provide sample \href{https://github.com/OpenBioSim/biosimspace_tutorials/blob/main/03_steered_md/scripts/sMD_slurm.sh}{slurm} and \href{https://github.com/OpenBioSim/biosimspace_tutorials/blob/main/03_steered_md/scripts/sMD_LSF.sh}{LSF} submission scripts to deploy the BSS steered MD scripts in different HPC environments. 

\subsubsection{sMD trajectory analysis}

The second notebook of this tutorial \href{https://github.com/OpenBioSim/biosimspace_tutorials/blob/main/03_steered_md/02_trajectory_analysis.ipynb}{02-trajectory-analysis} describes how to analyze data generated by a steered MD run. As the sMD simulation is run, the CV values are saved to a \textbf{COLVAR} file. It can be plotted to assess whether the sMD simulation has been successful. An example is shown in Figure \ref{fig:rmsd}.

\begin{figure}[htp]
    \centering
    \includegraphics[width=\linewidth]{LIVECOMS/03_steered_md/COLVAR_all.png}
    \caption{Evolution of Collective Variables throughout an sMD simulation. A) RMSD CV measuring distance to closed WPD loop conformation. B) Torsional angle CV measuring the Chi1 angle of residue Tyr 152. C)  CV measuring the distance between C$_{\gamma}$ atoms in residues Phe196 and Phe280. As the simulation progresses, the WPD loop RMSD is gradually lower (i.e. the loop is adopting a  closed loop conformation).}
    \label{fig:rmsd}
\end{figure}

The notebook also illustrates a "failed" steered MD trajectory, where the steering duration and force were insufficient to reach the target CV value.

\subsubsection{Example application - combining steered MD with Markov State Modeling}
While the information provided here focuses on running sMD simulations with BSS, there are multiple potential applications, such as studying membrane permeability~\cite{Wells2007} or ligand residence time\cite{Potterton2019}. Here we briefly highlight one application of sMD simulations enabled by BioSimSpace in the AMMo software project. AMMo ("Allostery in Markov Models") was developed to evaluate the allosteric effects of protein mutations or ligand binding by combining sMD with Markov State Models (MSMs). MSMs are used to give the probability of protein conformations and therefore can be used to model how a ligand affects the conformation ensemble of a target (e.g. whether the presence of a ligand decreases the active state probability and therefore is an allosteric inhibitor). 
 There is a lot to consider when building MSMs, and the method is not covered in this tutorial. AMMo uses the Python library \href{http://emma-project.org/latest/}{PyEMMA} was to implement MSMs. Extensive examples and documentation for PyEMMA are available elsewhere ~\cite{Wehmeyer_2019}. The integration of sMD with MSM in this allosteric modulation prediction workflow is illustrated in Figure \ref{fig:ensemble-protocol}. The notebook \href{https://github.com/OpenBioSim/BioSimSpaceTutorials/blob/main/03_steered_md/02_trajectory_analysis.ipynb}{02-trajectory-analysis} shows how a sMD trajectory can be sampled to extract a range of protein conformations suitable for inputs to an MSM workflow. Hardie et al. report a detailed study of allosteric modulators of PTP1B using this sMD/MSM methodology ~\cite{Hardie2023} and notebooks for the PTP1B case study are available on the \href{https://github.com/michellab/AMMo/tree/main/examples/ptp1b}{GitHub} page for AMMo.

\begin{figure}[htp]
\includegraphics[width=\linewidth]{LIVECOMS/03_steered_md/ensemble-md-protocol.png}
\caption{The steps used for enhanced sampling methods to gather data for statistical analysis of protein conformation ensemble. (1) Run steered MD along some collective variable (CV); (2) Extract snapshots that evenly sample available conformational space; (3) Run equilibrium MD simulations using extracted coordinates as seeds; (4) construct an MSM using trajectory data from step 3.}
\label{fig:ensemble-protocol}
\end{figure}