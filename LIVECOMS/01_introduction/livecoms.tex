This tutorial consists of five separate notebooks. 
\\
The first notebook \href{https://github.com/OpenBioSim/biosimspace_tutorials/blob/main/01_introduction/01_introduction.ipynb}{01-introduction} describes what functionality is currently available in BioSimSpace; what are the key concepts behind BioSimSpace, such as the use of a Sire system object to describe a generic molecular system; how interoperability between packages is achieved through a library of file converters;  what steps are taken to preserve the topology of a molecular system after it has been passed to various third-party tools. 
\\
%\input{LIVECOMS/01_introduction/02_molecular_setup}
The second notebook \href{https://github.com/OpenBioSim/biosimspace_tutorials/blob/main/01_introduction/02_molecular_setup.ipynb}{02-molecularsetup} 
provides an example of how to use BioSimSpace to set up a molecular system ready for simulation. Starting from a molecular topology in the form of a Protein Data Bank format file, the notebook teaches how to parameterize molecules using different molecular force fields, then solvate them using various water models before exporting the solvated topologies in a chosen file format. 
\\
%\input{LIVECOMS/01_introduction/03_molecular_dynamics}
The third notebook \href{https://github.com/OpenBioSim/biosimspace_tutorials/blob/main/01_introduction/03_molecular_dynamics.ipynb}{03-molecular-dynamics} describes how to use BioSimSpace to configure and run some basic molecular dynamics simulations. This notebook introduces the concept of \href{https://biosimspace.org/api/index_Protocol.html}{BioSimSpace.Protocol} to codify a shareable, re-usable and extensible simulation protocol. The concept of \href{https://biosimspace.org/api/index_Process.html}{BioSimSpace.Process}
is then introduced to provide functionality for configuring and running processes with several common molecular dynamics engines.  The process is then illustrated with code that implements interactive molecular dynamics simulations in the notebook. Finally, \href{https://biosimspace.openbiosim.org/api/index_Trajectory.html}{BioSimSpace.Trajectory} is introduced as a wrapper for the python software MDTraj and MDAnalysis to facilitate trajectory analyzes. 
\\
%\input{LIVECOMS/01_introduction/04_writing_nodes}
The fourth notebook \href{https://github.com/OpenBioSim/biosimspace_tutorials/blob/main/01_introduction/04_writing_nodes.ipynb}{04-writing-nodes} introduces the concept of nodes as interoperable workflow components. Nodes are robust and portable Python scripts that typically do a small, well-defined piece of work. All inputs and outputs from the node are validated and the node is written in such a way that it is independent of the underlying software packages, i.e. the same script can work with a range of different packages. In addition, nodes are aware of the environment in which they are run, so can be used interactively, from the command-line, or within a workflow engine. 
\\
%\input{LIVECOMS/01_introduction/05_running_nodes}
The fifth notebook \href{https://github.com/OpenBioSim/biosimspace_tutorials/blob/main/01_introduction/05_running_nodes.ipynb}{05-running-nodes} describes how BioSimSpace nodes can be exported as regular python scripts and executed in a variety of environments. This allows users to prototype BioSimSpace nodes in an interactive Jupyter notebook, and then deploy the same script without having to insert additional code. BioSimSpace nodes can be currently called from the command line and can also be imported within a BioSimSpace script, enabling the construction of complex workflows by reusing libraries of nodes. Nodes can also generate \href{https://www.commonwl.org}{common workflow language} wrappers, allowing BSS nodes to be plugged into any workflow engine that supports this standard. 
\\
