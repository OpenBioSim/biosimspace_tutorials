%%%%%%%%%%%%%%%%%%%%%%%%%%%%%%%%%%%%%%%%%%%%%%%%%%%%%%%%%%%%%%%%%%%%%%%%%%%%%%%%%%%%%%%%%%%
%%%%%%%%%%%%%%%%%%%%%%%%%%%%%%%%%%%%%%%%%%%%%%%%%%%%%%%%%%%%%%%%%%%%%%%%%%%%%%%%%%%%%%%%%%%
%%%%%%%%%%%%%%%%%%%%%%%%%%%%%%%%%%%%%%%%%%%%%%%%%%%%%%%%%%%%%%%%%%%%%%%%%%%%%%%%%%%%%%%%%%%
%%%%%%%%%%%%%%%%%%%%%%%%%%%%%%%%%%%%%%%%%%%%%%%%%%%%%%%%%%%%%%%%%%%%%%%%%%%%%%%%%%%%%%%%%%%
%%%%%%%%%%%%%%%%%%%%%%%%%%%%%%%%%%%%%%%%%%%%%%%%%%%%%%%%%%%%%%%%%%%%%%%%%%%%%%%%%%%%%%%%%%%
%%%%%%%%%%%%%%%%%%%%%%%%%%.  WRITE YOUR CHAPTER CONTENTS BELOW.  %%%%%%%%%%%%%%%%%%%%%%%%%%
%%%%%%%%%%%%%%%%%%%%%%%%%%%%%%%%%%%%%%%%%%%%%%%%%%%%%%%%%%%%%%%%%%%%%%%%%%%%%%%%%%%%%%%%%%%
%%%%%%%%%%%%%%%%%%%%%%%%%%%%%%%%%%%%%%%%%%%%%%%%%%%%%%%%%%%%%%%%%%%%%%%%%%%%%%%%%%%%%%%%%%%
%%%%%%%%%%%%%%%%%%%%%%%%%%%%%%%%%%%%%%%%%%%%%%%%%%%%%%%%%%%%%%%%%%%%%%%%%%%%%%%%%%%%%%%%%%%

\subsubsection{Introduction}


Computational chemists can support structure-activity relationship
studies in medicinal chemistry by making computer models that can
predict binding affinity of ligands to proteins. One of the most popular
techniques for this is Free Energy Perturbation (FEP), which relies on simulation alchemical transformations of ligands. Some
introductory reading is recommended. \cite{mey2020best, cournia_allen_sherman_2017, kuhn_firth-clark_tosco_mey_mackey_michel_2020}

This tutorial covers the basic principles of alchemical free energy calculations with BioSimSpace; how to setup, simulate and analyse Relative Binding Free Energy (RBFE) calculations for congeneric series of protein-ligand complexes; how to setup and analyse Absolute Binding Free Energy (ABFE) calculations of a ligand bound to a protein.  
The notebooks prompts the readers to complete a series of exercises that typically involve completing cells to test understanding of the material presented. 

\subsubsection{Setting up FEP calculations using BioSimSpace}

The first notebook of this tutorial \href{https://github.com/OpenBioSim/BioSimSpaceTutorials/blob/main/04_fep/01_intro_to_alchemy/alchemical_introduction.ipynb}{alchemical-introduction} introduces the basic functionality available in BioSimSpace to implement FEP calculations. 
The notebook describes first the functionality of the \href{https://biosimspace.openbiosim.org/api/index_Align.html}{BSS.Align} module. BSS.Align implements a variety of mapping algorithms based on maximum common substructure searches between a supplied pair of molecules to generate a 'merged' molecule used to describe an alchemical transformation. This is used to generate a merged molecule that describes the alchemical transformation of ethane into methanol.
Next the use of \href{https://biosimspace.openbiosim.org/api/generated/BioSimSpace.Protocol.FreeEnergy.html#BioSimSpace.Protocol.FreeEnergy}{BSS.Protocol.FreeEnergy} is described to setup the calculation of the relative hydration free energy of ethanol to methane using the FEP engines SOMD or GROMACS. 
The FEP simulations can in principle be executed directly from the notebook as a set of independent MD simulations run serially, but this is often too slow to be practical. The next section describes protocols to submit calculations in parallel on HPC resources.
Finally the notebook describes the setup of the relative binding free energy calculation of benzene to o-xylene bound to the protein T4 lyzozyme mutant. 
\\
The notebook then describes the use of \href{https://biosimspace.openbiosim.org/api/generated/BioSimSpace.FreeEnergy.Relative.html#}{BSS.FreeEnergy.Relative} to process completed FEP simulations for two legs of a thermodynamic cycle using the MBAR or TI algorithms. The resulting free energy changes are subtracted to yield a relative binding free energy for o-xylene to benzene. Finally, the plotting of overlap matrices is demonstrated to assess the reliability of a computed free energy change. The functionality covered by this notebook is illustrated by figure \ref{thermodynamic_cycle_fig}.

\begin{figure}[htp]
\includegraphics[width=\linewidth]{LIVECOMS/04_fep/introfep_updated.png}
\caption{ \textbf{A)} Thermodynamic cycle used to compute the relative hydration free energy of ethane to methanol. \textbf{B)} Visualisation of the atom mappings between ethane and methanol generated by BSS.Align \textbf{C)} Visualisation of overlap matrix generated by BSS.Notebook.plotOverlapMatrix to help assess the reliability of a calculated free energy change.}
\label{thermodynamic_cycle_fig}
\end{figure}


\subsubsection{RBFE calculation pipelines}

The second notebook of this tutorial \href{https://github.com/OpenBioSim/BioSimSpaceTutorials/blob/main/04_fep/02_RBFE/01_setup_rbfe.ipynb}{01-setup-rbfe.ipynb} describes the planning of a RBFE campaign for a congeneric series of protein-ligand complexes. This is illustrated using a dataset of ligands for the protein TYK2, taken from the benchmark set of Wang et al.\cite{Wang2015} 
First a drop-down menu is presented to the user to enable selection of a variety of configuration settings (such as the choice of forcefields to use for the ligands, the protein, or the FEP engine to select). In BSS 2023.3.0 the FEP engines SOMD and gmx (from the GROMACS software suite) are supported. An \href{https://github.com/michellab/BioSimSpace/tree/feature-amber-fep}{experimental feature branch} that implements support for RBFE calculations with the MD engine pmemd from the AMBER software suite is also available. 

Next \href{https://biosimspace.openbiosim.org/api/generated/BioSimSpace.Align.generateNetwork.html}{BSS.Align.generateNetwork} is used to interface to the LOMAP software~\cite{Liu2013} to propose a network of relative transformations that span the provided ligand dataset. The notebook describes interactive plotting of the proposed network, and how to make manual adjustments such as adding or deleting edges, or inserting new ligands in the network. Once the user is satisfied with the chosen FEP network, setup instructions are saved to disk. 
\\
Processing the entire TYK2 dataset involves setting up and running several hundred MD simulations of solvated ligands and protein-ligand complexes. This would be impractically slow if executed from a notebook on a single workstation. For convenience we provide a sample \href{https://github.com/OpenBioSim/BioSimSpaceTutorials/blob/main/04_fep/02_RBFE/scripts/run_all_slurm.sh}{slurm submission script} that processes the setup, simulation and analysis of the entire network constructed by the RBFE setup notebook on an HPC environment. This script may be adjusted for deployment on different slurm clusters, or as reference for implementation of the execution model on different schedulers. The functionality covered by this notebook is illustrated by figure \ref{rbfe_setup_fig}.
\\

\begin{figure}[htp]
\includegraphics[width=\linewidth]{LIVECOMS/04_fep/rbfe-setup.png}
\caption{ \textbf{A)} Schematic of the FEP pipeline in this tutorial. Whereas blue boxes represent notebooks run on a local machine, orange boxes represent python scripts run sequentially on a computing cluster.
\textbf{B)} Perturbation network proposed by BSS.Align.generateNetwork for a dataset of TYK2 ligands.} 
\label{rbfe_setup_fig}
\end{figure}


The third notebook \href{https://github.com/OpenBioSim/BioSimSpaceTutorials/blob/main/04_fep/02_RBFE/02_analysis_rbfe.ipynb}{02-analysis-rbfe} describes the analysis of the processed FEP network. 
The FEP network is first visualised using the utility NetworkX. Next mean relative binding free energies are evaluated for each edge by averaging the results from all replicates available for each edge. The standard error of the mean is used as an estimate of the statistical uncertainty of each edge RBFE. 
A scatter plot comparing relative vs calculated binding free energies is produced, this is only possible because experimental data is available for this dataset. 
\\
Next the set of RBFEs is converted into  a set of binding free energies with an arbitrary reference value using the cinnabar software to produce a scatter plot of calculated vs experimental binding free energies, together with statistical measures of accuracy (mean unsigned error, root mean squared error, Pearson coefficient and Kendall tau coefficient). The functionality covered by this notebook is illustrated by figure \ref{rbfe_analysis_fig}.
\\

\begin{figure}[htp]
\includegraphics[width=\linewidth]{LIVECOMS/04_fep/rbfe-analysis.png}
\caption{ \textbf{A)} Scatter plot of experimental vs computed pairwise $\Delta \Delta G_{bind}$ free energies from the analysed perturbation network. \textbf{B)} Scatter plot of $\Delta G_{bind}$ estimates after processing of the network. } 
\label{rbfe_analysis_fig}
\end{figure}

\subsubsection{ABFE calculations}
%
The fourth notebook \href{https://github.com/OpenBioSim/BioSimSpaceTutorials/blob/main/04_fep/03_ABFE/01_setup_abfe.ipynb}{01-setup-abfe} describes how to set up alchemical absolute binding free energy (ABFE) calculations with BSS for either GROMACS or SOMD. The free energy functionality is implemented in Exscientia's \textit{sandpit} area of this version of BioSimSpace. Sandpits are used to test experimental features without accidentally breaking core functionality of the toolkit. 

While RBFE calculations can be very useful in drug discovery, several important problems lie outside the scope of standard RBFE calculations. These include: calculating the binding free energies of structurally dissimilar ligands to a common target; calculating the binding free energies of the same ligand to the same protein with different binding poses; calculating the  binding free energies of a single ligand to a range of targets, as would be required to optimise selectivity or promiscuity. These quantities can be calculated using alchemical ABFE calculations because they utilise a more general ``double decoupling'' thermodynamic cycle than in RBFE calculations.\cite{gilson_statistical-thermodynamic_1997} This involves entirely removing the ligand's intermolecular interactions in the presence of restraints between the ligand and receptor, as shown in figure \ref{abfe_fig}A. 

\begin{figure}[htp]
\includegraphics[width=\linewidth]{LIVECOMS/04_fep/abfe-tutorial.png}
\caption{ \textbf{A)} Thermodynamic cycle for alchemical absolute binding free energy calculations. \textbf{B)} Sample convergence plots for legs 5 and 6 of the ABFE thermodynamic cycle for the MIF/MIF180 complex, showing especially poor convergence for the bound vanish stage prior to discarding initial non-equilibrated samples.} 
\label{abfe_fig}
\end{figure}

The notebook describes the use of \textit{BSS.Align.decouple} to tag a molecule in a BSS system that should have all its intermolecular interactions removed to compute its absolute binding free energy. This is illustrated here with the use of the ligand MIF180 bound to the protein MIF.\cite{Qian2019, Clark2023} 

Next, it is shown how to run and analyse a simulation of the fully-interacting protein-ligand complex to generate the required intermolecular restraints. Both actions are handled using \textit{BSS.FreeEnergy.RestraintSearch}. Currently, only the common Boresch restraints are supported,\cite{boresch_absolute_2003} but we plan to integrate multiple distance restraints in the future.\cite{Clark2023} 
The notebook illustrates the automated generation of Boresch restraints via two different implementations available in the package MDRestraintsGenerator,\cite{alibay_evaluating_2022, alibay_ialibaymdrestraintsgenerator_2021} or natively in BSS. Both methods aim to select stable restraints which mimic strong native receptor-ligand interactions. The notebook includes visualisation of the chosen restraints before free energy inputs are prepared, which allows the user to identify cases where the selected restraints may not be optimal, or where symmetry corrections may be required.\cite{duboue-dijon_building_2021} The notebook then demonstrates the use of \textit{BSS.FreeEnergy.AlchemicalFreeEnergy} to generate input files for the engines SOMD or GROMACS. This class in the Exscientia sandpit contains all of the functionality from \textit{BSS.FreeEnergy.Relative} in the main version of the code, as well as additional functionality required for ABFE calculations.

Since ABFE calculations can be time consuming we recommend parallel execution of the different legs of the thermodynamic cycle using an approach similar to that used in the RBFE tutorial.
\\

The fifth notebook \href{https://github.com/OpenBioSim/BioSimSpaceTutorials/blob/main/04_fep/03_ABFE/02_analysis_abfe.ipynb}{02-analysis-abfe} describes how to analyse an ABFE calculation to estimate the free energy of binding of a ligand. Sample output simulation data are provided for each leg of the double decoupling thermodynamic cycle and analysed using \textit{BSS.FreeEnergy.AlchemicalFreeEnergy.Analyse} to plot potentials of mean forces. The standard free energy of binding is then obtained by summing the free energy changes from each leg and adding a standard state correction term for the use of Boresch restraints, along with any symmetry corrections required. The notebook also describes how to carry out convergence analyses (see figure \ref{abfe_fig}B) to assess the robustness of the ABFE estimates.  
